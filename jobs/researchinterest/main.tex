% --------------------------------------------------------------
% Abhi's Standard math Preamble.
% --------------------------------------------------------------
 
\documentclass[12pt]{article}
 
%Math Packages
\usepackage[margin=1in]{geometry} 
\usepackage{amsmath,amsthm,amssymb,mathrsfs,bm}
\usepackage{mathtools}
\usepackage{xfrac}
%\usepackage{commath}
%\usepackage{minted}
\usepackage{enumerate}
\usepackage{fancyhdr}
\usepackage{hyperref}
 
\pagestyle{fancy}
\fancyhf{}
\rhead{Abhijit Chowdhary \thepage}

%Fonts
%\usepackage[sc,osf]{mathpazo}   % With old-style figures and real smallcaps.
%\usepackage{lmodern}
%\usepackage{eulervm}

\DeclareMathOperator{\sign}{sgn}

%Math commands
% Mathematical Fields
\newcommand{\R}{\mathbb{R}}
\newcommand{\N}{\mathbb{N}}
\newcommand{\Q}{\mathbb{Q}}
\newcommand{\Z}{\mathbb{Z}}
\newcommand{\C}{\mathbb{C}}
%Divides and Not Divides
\newcommand{\divides}{\bigm|}
\newcommand{\ndivides}{%
  \mathrel{\mkern.5mu % small adjustment
    % superimpose \nmid to \big|
    \ooalign{\hidewidth$\big|$\hidewidth\cr$\nmid$\cr}%
  }%
}
%Limit as n -> inf
\newcommand{\limty}{\lim_{n \to \infty}}
%sum from n=1 \to inf
\newcommand{\sumty}{\sum_{n = 1}^\infty}

\newcommand{\lindelof}{Lindel\={o}f }

\newtheorem*{lemma*}{Lemma}
 
\begin{document}
 
% --------------------------------------------------------------
%                         Start here
% --------------------------------------------------------------

 
\title{Research Statement}%replace X with the appropriate number
\author{Abhijit Chowdhary\\ %replace with your name
} %if necessary, replace with your course title

\maketitle

\section{Introduction}

I'm largely interested in Computational Mathematics, specificially regarding the
algorithms and analysis of them. Therefore, some fields that I find interesting
are numerical linear algebra, computational group theory, fast PDE/ODE solvers,
etc. In addition, I have a substantial interest in the methods used to
accomplish such tasks, such as compilers and high performance computing. Most
have my study has gone to the pursuit of such topics, and most of my projects
are directly related.

Ideally, I would like to work on projects or libraries similar to FLINT
\footnote{Fast Library for Number Theory: http://www.flintlib.org/}, LinBox
\footnote{Exact computational linear algebra: http://www.linalg.org/}, or FEniCS
\footnote{Finite Element Computational Software: https://fenicsproject.org/}, as
they do exactly what I'm interested in. However, I am trained in the relevant
pure fields, and perhaps can add value to a project here by taking a
computational approach. 


\section{Projects}

To give you a better idea of what I've done so far, here's a few applications.

\subsection{FFPoly}

This is a new project I've begun, and have taken into my high
performance computing course as a term project.  In my graduate algebra course I
was exposed to some algebraic number theory, and upon request, my professor
showed me a few papers regarding the computational aspects of it. Intrigued, I
decided to write a module in C\textit{++} implementing a element of the field
$\mathbb{F}_p[x]$, for small $p \in \mathbb{Z}$. 

This polynomial field is interesting, since the splitting behavior of minimal 
polynomials of number fields gives information of the splitting behavior of the
corresponding primes in the algebraic number field. Furthermore, using
Minkowski's bound, we could said information to verify the class groups of
certain small quadratic number fields, which is interesting. SageMath has such a
capability, but I had a couple of special ideas that I wanted to try;
specifically I wanted to investigate the performance of parallelization on these
problems. Regarding this, there is lots of potential for optimization in the
field operations and factoring steps, and some interesting active research going
on in the field \footnote{See the paper \textit{Parallel Integer Polynomail
Multiplication} by Chen, Covanov, Mansourim Maza, Xie and Xie}. The goal here
is to research a few ways to turn some computational algebra algorithms into
parallel friendly methods, and use them to maybe try and beat out a few similar
serial libraries in speed.

\subsection{Algebraic Point Set Surfaces}

During the Spring 2018 semester, my sophomore year, I took this course called
Geometric Modeling. This was a masters/PhD level course, intended for those
interested in computer graphics research or cutting edge level industry work,
and I attended on reccomendation from a professor who thought the coursework was
wild. While the whole class was a beautiful exercise in Numerical Optimization
and Linear Algebra, my term project is currently a main motivation for me to
pursue research in this field.

My final project was to implement the paper \textit{Algebraic Point Set
Surfaces} by Gunnebaud and Gross. Roughly speaking, in geometric modeling we
have the pipeline: Physical object $\to$ spatial point cloud representing
external surface $\to$ triangularized mesh representing object. The paper
focused on a fast method to take the point cloud to a triangularized mesh. A 
prevailing method was to consider an open neighborhood of a point in the cloud,
and fit a plane to it that minimized the distance to the points in that
neighborhood. Such a plane would become part of the later mesh. The paper
suggested fitting a sphere to these points instead, which was hugely
advantageous in regions of high curvature. Such a fit would result in a
non-linear optimization problem, but the key idea of using the algebraic
definition of a sphere, results in a nicer computation. Specifically, given
normals we can have a linear system solve, and the also proposed a method to
estimate said normals with a generalized eigenproblem.

The truly difficult parts of this project to me wasn't the actual graphics
theory, that was laid out nicely in the paper. What really plagued me was the
implementation; I ran into serious runtime issues in the actual construction of
the matrices involved in the eigenproblem, and later the system solve. I spent
most of my time making various vectorization arguments to speed up the
construction, and then later making sparsity arguments in the actual system
solve. 

Despite the difficulty of my first research level project, I truly enjoyed the
construction process, nothing to say of the relief when I finally managed to
triangularize a bunny \footnote{A classic point cloud to construct is the
Stanford Bunny model.}. In addition, it verified to me that I not only enjoyed
the study of numerical methods, but liked and was capable of research and
development in them.

\end{document}
