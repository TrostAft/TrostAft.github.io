% resume.tex
%
% (c) 2002 Matthew Boedicker <mboedick@mboedick.org> (original author) http://mboedick.org
% (c) 2003-2007 David J. Grant <davidgrant-at-gmail.com> http://www.davidgrant.ca
%
% This work is licensed under the Creative Commons Attribution-ShareAlike 3.0 Unported License. To view a copy of this license, visit http://creativecommons.org/licenses/by-sa/3.0/ or send a letter to Creative Commons, 171 Second Street, Suite 300, San Francisco, California, 94105, USA.

\documentclass[letterpaper,11pt]{article}

\newlength{\outerbordwidth}
\pagestyle{empty}
\raggedbottom
\raggedright
\usepackage[svgnames]{xcolor}
\usepackage{framed}
\usepackage{tocloft}
\usepackage{amssymb}
\usepackage{etoolbox}
\usepackage{hyperref}
\robustify\cftdotfill

%-----------------------------------------------------------
%Edit these values as you see fit
\setlength{\outerbordwidth}{3pt}  % Width of border outside of title bars
\definecolor{shadecolor}{gray}{0.75}  % Outer background color of title bars (0 = black, 1 = white)
\definecolor{shadecolorB}{gray}{0.93}  % Inner background color of title bars

%-----------------------------------------------------------
%Margin setup
\setlength{\evensidemargin}{-0.25in}
\setlength{\headheight}{-0.25in}
\setlength{\headsep}{0in}
\setlength{\oddsidemargin}{-0.25in}
\setlength{\paperheight}{11in}
\setlength{\paperwidth}{8.5in}
\setlength{\tabcolsep}{0in}
\setlength{\textheight}{9.75in}
\setlength{\textwidth}{7in}
\setlength{\topmargin}{-0.3in}
\setlength{\topskip}{0in}
\setlength{\voffset}{0.1in}

%-----------------------------------------------------------
%Custom commands
\newcommand{\resitem}[1]{\item #1 \vspace{-2pt}}
\newcommand{\resheading}[1]{\vspace{8pt}
  \parbox{\textwidth}{\setlength{\FrameSep}{\outerbordwidth}
    \begin{shaded}

\setlength{\fboxsep}{0pt}\framebox[\textwidth][l]{\setlength{\fboxsep}{4pt}\fcolorbox{shadecolorB}{shadecolorB}{\textbf{\sffamily{\mbox{~}\makebox[6.762in][l]{\large #1} \vphantom{p\^{E}}}}}}
    \end{shaded}
  }\vspace{-5pt}
}

\newcommand{\ressubheading}[4]{
\begin{tabular*}{6.5in}{l@{\cftdotfill{\cftsecdotsep}\extracolsep{\fill}}r}
		\textbf{#1} & #2 \\
		\textit{#3} & \textit{#4} \\
\end{tabular*}\vspace{-6pt}}

\begin{document}

\begin{tabular*}{7in}{l@{\extracolsep{\fill}}r}
\textbf{\Large Abhijit Chowdhary}  & (240)-715-8308\\
369 Broome Street, Apartment 6 &  abhijit9331@gmail.com \\
New York, NY & \url{https://abhijit-c.github.io/}\\
\end{tabular*}
\\

\vspace{0.1in}

\resheading{Education}
\begin{itemize}
\item
	\ressubheading{New York University}{New York, NY}{B.A. Joint Mathematics and
    Computer Science (In Major GPA: 3.657)}{Sep. 2016 - Present}
	\begin{itemize}
		\resitem{Relevant undergraduate courses: 
      Algorithms, 
      Chaos and Dynamical Systems,
      Computer System Organization, 
      Operating Systems,
      Honors Algebra I/II, 
      Honors Analysis I/II, 
      Honors Linear Algebra, 
      Honors Probability Theory, 
      Numerical Computation, 
      Topology.}
    \resitem{Relevant graduate/PhD courses: 
      Algebra, 
      Finite Element Method,
      Fundamental Algorithms, 
      Geometric Modeling, 
      High Performance Computing,
      Methods of Applied Math, 
      Numerical Methods I/II, 
      Partial Differential Equations.}
	\end{itemize}

\item
	\ressubheading{University of Maryland}{College Park, MD}{Visiting
    Student (GPA: 3.925)}{Summer 2017,2018}
	\begin{itemize}
		\resitem{Coursework: Complex analysis, Number theory, Partial
        Differential Equations, Introduction to Artificial Intelligence.}
	\end{itemize}

\end{itemize}

\resheading{Projects and Activities}
\begin{itemize}

\item
	\ressubheading{Math REU: Imperfect Periodic Patterns}{Athens, OH}{Ohio
  University}{June 2019 - August 2019}
	\begin{itemize}
    \resitem{I joined a research team under professor Qiliang Wu, and another
      undergraduate Mason Haberle from Berkeley in researching the field of
      pattern formation.}
    \resitem{Our team specifically set out to prove nonlinear stability of the
      2D Swift-Hohenberg equation at the zigzag boundary, and as of now we've
      completed the proof and the paper is in the draft stages.}
    \resitem{The challenge here was mostly on how to adapt known techniques
      first to the Swift-Hohenberg equation, and second to higher dimensions.
      This was mostly a conceptual difficulty in the functional analysis
      framework surrounding the current research which we had to resolve.}
	\end{itemize}

\item
	\ressubheading{Tutor and TA at Courant}{New York, NY}{NYU
    }{Sep. 2017 - Present}
	\begin{itemize}
		\resitem{I work at a Tutor and TA to Professor Siegel at NYU for his
        undergraduate basic algorithms and graduate fundamental algorithms
        course.}
        \resitem{I host tutoring sessions for students to come in and answer
        questions, and I help to build course materials.}
	\end{itemize}

\item
	\ressubheading{Algebraic Point Set Surfaces Implementation}{}{Geometric
    Modeling (Grad)}{Apr. 2018 - May. 2018}
	\begin{itemize}
		\resitem{For a class final project, I implemented the theory in the
        paper \textit{Algebraic Point Set Surfaces} by Ga\={e}l Gunnebaud and
        Markus Gross from ETH Zurich.}
        \resitem{The Paper presented an alternative method to take a point cloud
        to a triangularized mesh, and another method to estimate normals from a
        point cloud using algebraic fitting of a sphere.}
		\resitem{This was mostly a challenge in comprehension of the paper and
		implementation, notably fighting with Eigen to try and constuct and
		solve the sytems in an efficient manner.}
        \resitem{Heavy use of the libraries Eigen, libigl, and nanoflann.
        Written in C++.}
	\end{itemize}

\item
  \ressubheading{Parareal}{}{High Performance Computing and Numerical Methods
  II (Grad)}{Apr. 2019 - May 2019}
	\begin{itemize}
    \resitem{For a class final project, I decided to look into parallel
    techniques for solving ordinary differential equations, and something that
    caught my eye was the parallel-in-time algorithm, \textit{Parareal}.}
    \resitem{For this project, I implemented and analyzed this algorithm, and
    further tested it's scaling properties on the HPC cluster Prince here at
    NYU.}
    \resitem{Heavy use of Eigen and OpenMP, it's a header only library. Written
    in C++}
    \resitem{You can see the project here: https://github.com/abhijit-c/Parareal}
	\end{itemize}

\item
	\ressubheading{ffpoly}{}{Personal Project}{Dec. 2018 - Present}
	\begin{itemize}
		\resitem{After learning some algebraic number theory in my graduate
        algebra course, I decided to code an implementation of elements of the
        polynomial field $\mathbb{F}_n[x]$.}
        \resitem{Still in the infancy stages of the project and learning some of
        the theory in computational algebraic number theory as I code.}
        \resitem{You can see the project here:
        https://github.com/abhijit-c/ffpoly}
	\end{itemize}

\item
	\ressubheading{Project Euler}{}{Personal Hobby}{2016 - Present}
	\begin{itemize}
		\resitem{Participant in the Project Euler mathematical programming
        challenges.}
        \resitem{Have solved 60 problems (Top 3.8\% as of Jan 2019)}
	\end{itemize}

\item
	\ressubheading{First Robotics Team Member, Team 2849: Ursa Major}{Columbia, MD}{Hammond High
    School}{Sep. 2012 - Present}
	\begin{itemize}
		\resitem{A robotics team; every new year they gather for a challenge
        created by FIRST Robotics to build a robot in six weeks.}
        \resitem{I worked as a build-team / programming-team flex member and
        team captain during my student years, and now I help as a programming
        and design mentor during their season.}
        \resitem{Has managed to consistantly reach eliminantion and championship
        rounds at the regional level.}
        \resitem{See their github here: https://github.com/teamursamajor}
	\end{itemize}

\end{itemize}

\resheading{Skills}

\begin{description}
\item[Programming and Markup Languages:]
C, C++, Python, Matlab, Mathematica, \LaTeX.
\item[Languages:]
English, Latin, Broken Hindi
\item[Operating Systems:]
Linux (Arch Linux laptop, Debian desktop)
\item[Minor Mechanical Fabrication Skills]
\end{description}

\resheading{Interests}

\begin{description}
\item[Academic:] Numerical Methods and Algorithms, High Performance Computing,
  Dynamical Systems, Approximation of the solutions to PDEs, Computational
  Algebra and Number Theory. 
\item[ACM:] Student member of ACM and in the EBoard of the ACM Chapter of NYU.
\item[Computers:] Have built and maintained my PC since 2012, and have
modded my thinkpad X230 with various screen and hardware upgrades. Fulltime
Linux enthusiast since 2012.
\end{description}

\end{document}
